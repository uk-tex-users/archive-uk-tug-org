\documentclass[11pt,a4paper]{article}


\makeatletter
\pagestyle{empty}
\@twosidetrue
\flushbottom
\frenchspacing
\textwidth 170.5mm
\textheight 249mm
%\textheight 229mm
\columnwidth 83.25mm

%Trim sizes
\setlength\voffset{-1in}
\setlength\hoffset{-1in}
\topmargin 10mm
\setlength\oddsidemargin{18mm}%back margin on odd pages
\setlength\evensidemargin{21.5mm}%fore margin on even pages
\setlength\paperwidth{210mm}
\setlength\paperheight{297mm}
%Needed to set PDF page size
\special{papersize=210mm,297mm}
\makeatother


\begin{document}

\section*{UK \TeX\ Users Group --- Chair's report --- 16 October 2010}


I have been Chair of the UK TeX Users Group for four years now and am
not standing for re-election.  Although not a founding member of UKTUG
I joined it in its first year (1990), have served on the committee for
several years, and in particular organised several effective and well
attended meetings and written many articles for our journal
Baskerville during the 1990s.

In this report, in addition to recent news, I will take a longer view.
In 1990 there were 12.4 million mobile (then called cellular) phones.
Last year there were approximately 4.6 billion, 370 times as many.  In
1989 the USA had its first commercial dial-up access internet service
provider, and in 1992 Congress allowed the National Science Foundation
funded network to interconnect with commercial networks.  In 1990
there were 313,000 internet host computers.  In 2009 there were 681
million, which is 2,175 times as many.

In 1990 and 1991 Tim Berners-Lee started the World Wide Web while at
CERN.  In 1993 there was Mosaic, the first widespread graphical web
browser and approximately 600 web sites. By 1996 there were about
100,000 of which half were dot-com.  Also in 1996 two PhD students at
Stanford (Larry Page and Sergey Brin) started the research project
that became Google, which last year had a profit of \$6.5 billion from
a revenue of \$23.6 billion.

Facebook was launched in February 2004 and by July this year it had
over 500 million active users and estimated revenues of \$0.8 billion.
It is estimated that there are 6.9 billion people alive now, so about
1/14 are on Facebook and about 2/3 have a mobile phone.

In short, over the past 20 years humanity has built an electronic
communication network that reaches most of the globe and is used by
perhaps a majority of the world's population.  This system now
embraces person-to-person communication (as in the telephone),
broadcast communication (as in newspapers, radio and television) and
also distribution of books, film and recorded music.

All this is not possible without agreed behaviour, without standards.
The world's oldest international organisations are the Central
Commission for Navigation on the Rhine (1816), the International
Telecommunication Union (1869) and the Universal Postal Union (1874).
For example, prior to the UPU a letter sent abroad often needed stamps
of several countries on it.

In 1977, when Don Knuth starting working on \TeX, paper was by far the
dominant medium for written communication.  Books, letters, bills,
newspapers, timetables, tickets, advertisements, diaries, logbooks are
all examples.  Punched cards and paper were used for textile looms
(1725, Jacquard 1801), ticker tape (1870), 1890 US census (Hollerith)
and player pianos (flourished 1896--1930).  Hollerith was a founder of
what became IBM.  In 1977 paper was in libraries the dominant media
for data storage, along with vinyl for music.  Around then the Betamax
and VHS video tape formats were introduced.

At that time academic, scientific, government and commercial data
processing were major users of electronically stored written
information.  Now ordinary people are major users.  In 1980 IBM
produced the first gigabyte capacity hard drive, the size of a fridge,
550 lbs and \$40,000.  Today \pounds 50 will buy a 1 terabyte hard
drive, and \pounds 5 a 2 gigabyte USB drive.  The source
file \texttt{tex.web} for \TeX\ occupies about 1 megabyte.

Typesetting is an early example of this move from paper to digital
media.  Prior to the rise of phototypesetting (shining light through
negative images of characters onto photographic film) in the 1970s,
hot metal typesetting was often used to create a single original which
could be photographed and used to produce offset lithoplates.
Phototypesetting was, in turn, replaced by digital typesetters (in
important ways similar to modern laser printers) driven by a computer.

This was the situation when Don Knuth started working on \TeX\ in
1977.  There were computers and phototypesetters, and a software
gap. \TeX\ and Metafont admirably filled this gap, particularly for
mathematical content.  PostScript was developed by John Warnock and
released by Abode in 1982.  PDF followed in 1993.  Digital typesetting
is now taken for granted.  We generate our PDF file, send it to a
print supplier, who then returns thousands of printed copies.

Today many people prefer to receive written communication
electronically, as text or chat message, email, web-page or PDF.  Much
typeset material is read on-screen and is seldom printed.  And the
web-page is a major medium for written communication.  Last year
Google bought a disused paper mill in Finland, for conversion into a
data centre, at a total cost of \$260 million.  All for storage and
transmission of digital information.

Although paper is far from dead, this enormous shift from paper to
electronic media is of immense importance for the \TeX\ community.
Sadly, we are barely coping.  Translation of \LaTeX\ to XML and
vice-versa is not straightforward.  The problems of mathematical
content on web pages have hardly been solved.  Installation and
running of \TeX\ requires a long download and many technical skills.
The \LaTeX3 project, started in 1993, is still far from completion.

Scalable Vector Graphics (SVG) provides us with a new opportunity.  I
describe it as PDF for web pages, with some elements of Flash.
Although the W3C adopted SVG as a standard in 2001, it is not yet
widely used due to non-adoption by Microsoft.  But that is changing.
All modern browsers support SVG, including Internet Explorer 9, but
not IE7 and IE8.

SVG, together with web fonts, allow \TeX\ quality typesetting to be
displayed as a scalable part of a web page on many modern
browsers. For IE7 and IE8 emulations are available.  (For graphical
material Google's svgweb translates SVG in the browser to Flash.  For
typeset matter HTML, CSS and webfonts provide a better emulation.  The
MathJax software gives an excellent example of what can be done now.)

There are many important challenges and opportunities facing use,
besides SVG.  Improved documentation and training, translation to and
from XML, simplified installation, Unicode support are examples.  But
SVG is special for two reasons.  First, it gives us an opportunity
to establish \TeX\ as the definitive means of rendering mathematics
both for display on web pages and for print.  Second, SVG will become
the major medium for reading typeset material on web pages and
elsewhere.  For example, every EPUB reader must support SVG.

This, then, is my view of the past 20 years, which roughly speaking
encompasses the life of the UK \TeX\ Users Group, and of some of the
challenges facing us now.

In the past four years we have made steady progress.  When I became
Chair things were so bad that there was open talk of dissolving the
organisation.  We instituted a subscription holiday and set up a
projects fund to reduce our considerable surplus.  Broadly speaking
both have done well.  We have shown ourselves able to spend money on
supporting \TeX\ in the UK and more widely.  The AGM is being asked to
end the subscription holiday.

We have funded two projects in 2007--8.  We provided Jonathan
Kew \pounds 1,600 for the TeXworks integrated development
environment, which was completed and is now part of \TeX\
distributions.

We also provided \pounds 1,700 (with a second equal instalment on
receipt of a progress report) to add Unicode maths support to Latin
Modern and the TeXGyre font collection (a project led by Hans Hagen).
Here there seems to be no progress and no expenditure.

We have replaced our previous rather quirky constitution with
something that will serve us better. We have held some fairly
successful meetings, and earlier this year we organised LaTeX
training.

Particular thanks are due to David Saunders, who has been an excellent
Treasurer, a steady and reliable voice, and who led greatly on the new
Constitution; to Joseph Wright, who has ably managed our web-site,
handled membership and much administration, ran \LaTeX{} training with
Nicola Talbot (who is is also thanked); and to Jonathan Webley who
with much independent effort has restarted our magazine Baskerville.

I wish all new and continuing committee members and our new Chair,
Alun Moon, all the best for the coming years.


\smallskip

\rightline{Jonathan Fine}

\end{document}
