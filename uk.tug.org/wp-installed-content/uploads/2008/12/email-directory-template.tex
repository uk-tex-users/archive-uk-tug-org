% This is a UK-TUG Training Template
% http://uk.tug.org/

% ----------------------------------------------------------------------
% All LaTeX documents start with a \documentclass line. Here, we are
% loading the standard "article" class. The optional argument (in square
% brackets) makes sure the paper size is correct and sets the font size.
\documentclass[12pt,a4paper]{article}
% ----------------------------------------------------------------------

% Comment lines start with a % character. This text will be completely 
% ignored when compiled with LaTeX.

% Programs such as TeXworks will detect the following line, and will 
% automatically use pdfLaTeX to compile the document.
% 
% !TEX TS-program = pdflatex

% ----------------------------------------------------------------------
% The behaviour of LaTeX can be altered by loading "packages" using the
% \usepackage command. Each one gives us a particular feature to make
% life easier.
% ----------------------------------------------------------------------

% In TeX, fonts are "encoded", with each space in the font containing
% a different character. For most European users, the best setting is
% the "T1" (modern) system, which is set up next.
\usepackage[T1]{fontenc}

% Babel is a package to make working with different languages easier.
% LaTeX works in US English by default; in the UK, hyphenation is 
% slightly different, and so we load babel to sort this out.
\usepackage[english,UKenglish]{babel}

% The standard LaTeX font is Computer Modern. Here, we will load the 
% Courier font for monospaced material and the Palatino font for
% body text. Notice that the package names are given as a comma
% separated list.
\usepackage{courier,mathpazo}
% Other popular options are
% - New Century Schoolbook:  \usepackage{newcent}
% - Times Roman           :  \usepackage{mathptmx}

% The datatool package will allow us to get the data for merging
% from a CSV (comma separated value) file.
\usepackage{datatool}

% The longtable package is used to construct a table which can be longer
% than one page. This may need two or more LaTeX runs to get the column
% widths correct.
\usepackage{longtable}

% The hyperref package provides a number of series for hyperlinks and
% so on. Here, it is used to make the e-mail addresses clickable links.
\usepackage[colorlinks,urlcolor=black]{hyperref}

% Here, the file "directory.csv" is loaded as database "directory". It
% is possible to use more than one database with a LaTeX file.
\DTLloaddb{directory}{directory.csv}

% ----------------------------------------------------------------------
% Some settings about the file for the title.
% ----------------------------------------------------------------------
\title{Membership Directory}
\author{Some Group}
\date{2009}

% ----------------------------------------------------------------------
% All printed material must be inside \begin{document}.
% ----------------------------------------------------------------------
\begin{document}

% ----------------------------------------------------------------------
% Make the title.
% ----------------------------------------------------------------------
\maketitle

% ----------------------------------------------------------------------
% The information itself is placed in a table, which will align the
% material. There are two columns, both left-algined.
% ----------------------------------------------------------------------
\begin{longtable}{ll}
% The \DTLforeach macro loops over each row of the CSV file. The first
% argument is the name of the database to use. The second argument maps
% each database column to a TeX macro name.
  \DTLforeach{directory}{%
    \firstname =Forename,
    \lastname  =Surname,
    \email     =Email%
  }{%
% This third argument contains the text to appear for each row of the
% database. The fields are inserted using the macro names defined above.
% Thus \firstname, \lastname and \email insert the relevant part of the
% current row of the database here.
% 
% The \unskip macro is used to remove any extra spaces from \firstname,
% so that there is always exactly one space between the first and last 
% name.
% 
% The \href and \nolinkurl macros are from the hyperref package. \href
% creates a clickable link, with the first argument taking the link
% information and the second the text to print. To make sure that the
% e-mail address prints correctly the \nolinkurl macro is used to 
% handling characters such as "_".
    \firstname\unskip\ \lastname & 
      \href{mailto:\email}{\nolinkurl{\email}} 
      \\[1ex] % The table row has extra space added.
  }
\end{longtable}

\end{document}
