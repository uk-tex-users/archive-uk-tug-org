% ----------------------------------------------------------------------
% All LaTeX documents start with a \documentclass line. Here, we are
% loading the standard "article" class. The optional argument (in square
% brackets) makes sure the paper size is correct and sets the font size.
\documentclass[12pt,a4paper]{article}
% ----------------------------------------------------------------------

% Comment lines start with a % character. This text will be completely 
% ignored when compiled with LaTeX.

% Programs such as TeXworks will detect the following line, and will 
% automatically use pdfLaTeX to compile the document.
% 
% !TEX TS-program = pdflatex

% ----------------------------------------------------------------------
% The behaviour of LaTeX can be altered by loading "packages" using the
% \usepackage command. Each one gives us a particular feature to make
% life easier.
% ----------------------------------------------------------------------

% In TeX, fonts are "encoded", with each space in the font containing
% a different character. For most European users, the best setting is
% the "T1" (modern) system, which is set up next.
\usepackage[T1]{fontenc}

% Babel is a package to make working with different languages easier.
% LaTeX works in US English by default; in the UK, hyphenation is 
% slightly different, and so we load babel to sort this out.
\usepackage[english,UKenglish]{babel}

% The standard LaTeX font is Computer Modern. Here, we will load the Times
% New Roman font instead. 
\usepackage{mathptmx}
% Other popular options are
% - New Century Schoolbook: \usepackage{newcent}
% - Palatino:  			 \usepackage{mathpazo}

% The geometry package makes changing margins easy.  
\usepackage[margin=2cm]{geometry}

% ----------------------------------------------------------------------
% This information will be used by the \maketitle macro
% ----------------------------------------------------------------------
\author{Some Small Group}
\title{Receipts and Payments Account}
\date{2007--08} % Notice the use of "--" for an en-dash in the output

% ----------------------------------------------------------------------
% All printed material must be inside \begin{document}
% ----------------------------------------------------------------------
\begin{document}

% Create a title for the document using the information supplied above.
\maketitle

% Almost all of the real content of the page will be inside a tabular
% environment. For "formal" tables, vertical lines are usually 
% considered to be very bad design. However, they are appropriate here.
\begin{tabular}{lr|r|r||r} % l = left-align, r = right-align
% Column 1: Description
% Column 2: Unrestricted funds
% Column 3: Restricted funds
% Column 4: Total
% Column 5: Last year
% Each column is separated by a &, and each row ends with \\.#

% Using \parbox lets us put multiple lines of text in one cell. The 
% boxes have a fixed width, here given in "em" (a printers unit 
% dependant on the font in use).
 & \parbox[b]{5.5em}{\raggedleft Unrestricted \\ funds}
 & \parbox[b]{4.5em}{\raggedleft Restricted \\ funds}
 & \parbox[b]{2.5em}{\raggedleft Total}
 & \parbox[b]{4em}{\raggedleft Previous \\ year} \\
 
\textbf{RECEIPTS} \\[2ex] % The optional argument [2ex] here puts more  
                          % space between lines
 
UK Membership                & 10  &       & 10    & 40    \\
Joint membership with US-ORG &     & 1,000 & 1,000 & 1,140 \\
Donations                    & 100 &       & 100   & 40    \\[1ex]

\textbf{Sub-total}           & \textbf{110} & \textbf{1,000} 
  & \textbf{1,110} & \textbf{1,220} \\[1ex]
  
Bank interest                & 345 &       & 345   & 345   \\[1ex]
  
\textbf{Total receipts}       & \textbf{455} & \textbf{1,000} 
  & \textbf{1,455} & \textbf{1,565} \\[1ex]
 
\textbf{PAYMENTS} \\[2ex]

Joint membership with US-ORG &      & 960 & 960 & 1,440 \\
Exchange rate movement       & (15) & 15  &     &       \\
Some software fund           &      & 80  &  80 &    60 \\
Bank charges                 &   90 &     &  90 &    80 \\
AGM Costs                    &  310 &     & 310 &   250 \\
Printing                     &   20 &     &  20 &    30 \\
Web hosting                  &   25 &     &  25 &    30 \\
Accountancy                  &  200 &     & 200 &   180 \\

\textbf{Total payments}      & \textbf{630} & \textbf{1,055} 
  & \textbf{1,685} & \textbf{2,070} \\[1ex]
  
\\[1ex]

\textbf{Net receipts (payments)} & \textbf{(175)} & \textbf{(55)} 
  & \textbf{(230)} & \textbf{(505)} \\[2ex]
  
Cash funds previous year end & 15,955 & 810 & 16,765 & 17,040 \\[1ex]

\textbf{Cash funds this year end} & \textbf{15,780} & \textbf{755} 
& \textbf{16,535} & \textbf{16,765} \\

\\[2ex]

% The \multicolumn macro lets us use two or more columns merged
% together, or to change the alignment for a single cell. 
\textbf{Represented by}
 &  \multicolumn{2}{l}{Current account} & 6,785 &  6,765 \\
 &  \multicolumn{2}{l}{Saving account } & 9,750 & 10,000 \\

\end{tabular}

% Move down the page a bit.
\vspace{2cm}

\begin{flushright}
A.~N.~Other      \\
Treasurer        \\
31st August 2008 \\
\end{flushright}

\end{document}
