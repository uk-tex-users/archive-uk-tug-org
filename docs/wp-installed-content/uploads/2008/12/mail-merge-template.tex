% This is a UK-TUG Training Template
% http://uk.tug.org/

% ----------------------------------------------------------------------
% All LaTeX documents start with a \documentclass line. Here, we are
% loading the standard "letter" class. The optional argument (in square
% brackets) makes sure the paper size is correct and sets the font size.
\documentclass[12pt,a4paper]{letter}
% ----------------------------------------------------------------------

% Comment lines start with a % character. This text will be completely 
% ignored when compiled with LaTeX.

% Programs such as TeXworks will detect the following line, and will 
% automatically use pdfLaTeX to compile the document.
% 
% !TEX TS-program = pdflatex

% ----------------------------------------------------------------------
% The behaviour of LaTeX can be altered by loading "packages" using the
% \usepackage command. Each one gives us a particular feature to make
% life easier.
% ----------------------------------------------------------------------

% In TeX, fonts are "encoded", with each space in the font containing
% a different character. For most European users, the best setting is
% the "T1" (modern) system, which is set up next.
\usepackage[T1]{fontenc}

% Babel is a package to make working with different languages easier.
% LaTeX works in US English by default; in the UK, hyphenation is 
% slightly different, and so we load babel to sort this out.
\usepackage[english,UKenglish]{babel}

% The standard LaTeX font is Computer Modern. Here, we will load the New
% Century Schoolbook font instead. 
\usepackage{mathptmx}
% Other popular options are
% - New Century Schoolbook:  \usepackage{newcent}
% - Palatino              :  \usepackage{mathpazo}

% The csvtools package will allow us to get the data for merging
% from a CSV (comma separated value) file.
\usepackage{csvtools}

% To create envolpes easily.  The settings needed here depend on your 
% printer and PDF viewer. These settings work with HP laser printers
% printing from Acrobat.
\usepackage[dlenvelope,nocapaddress]{envlab}
\makelabels % Comment out this line if you do not want labels

% ----------------------------------------------------------------------
% Some settings about you and the letters: these will be used in all of
% the letters.
% ----------------------------------------------------------------------
\address{Your name \\ Your Street \\ Your Town \\ Your Postcode}
\date{\today} % Fill in the date you want here.
\name{Your name}
\signature{Your name for signature}

% ----------------------------------------------------------------------
% A customised command so that completely empty data items to not cause
% errors: if the field is empty, the output will be skipped.
% ----------------------------------------------------------------------
\makeatletter % We need to access some internal commands 
\newcommand*{\onlyifnotempty}[1]{% Create a new macro needing one value
  \ifx#1\empty
    \expandafter\@gobble % Empty input: ignore the next thing
  \else
    \expandafter\@firstofone % Use the next thing unchanged
  \fi
}
\makeatother

% ----------------------------------------------------------------------
% All printed material must be inside \begin{document}.
% ----------------------------------------------------------------------
\begin{document}

% ----------------------------------------------------------------------
% For the mail merge to work, everything is inside \applyCSVfile
% ----------------------------------------------------------------------
\applyCSVfile{example.csv}{% Using the accompanying example.csv database
% The database used looks like this:
% 
% Title,Firstname,Surname,AddressI,AddressII,AddressIII,Town,Postcode
% Miss,Alison,Smith,1 The Street,,,SomeTown,AB1 2XY
% Mr,Ben,Jones,2 The Close,SmallVillage,,OtherTown,AB2 4XY
% Mr,Chris,Brown,Housename,The Main Street,OddVillage,BigTown,AB3 6XY
% 
% (without the % symbols at the start of each line).

% Each item to be inserted is named as \insert<name-of-field>
\begin{letter}{
  \insertFirstname~\insertSurname \\
  \insertAddressI \\
  \onlyifnotempty{\insertAddressII}{\insertAddressII \\}%
  \onlyifnotempty{\insertAddressIII}{\insertAddressIII \\}%
  \insertTown \\
  \insertPostcode
}

% This letter layout is the standard one: a better format is shown in 
% the file letter-template.tex available at http://uk.tug.org

\opening{Dear \insertTitle~\insertSurname} % 

The letter text goes here. If the letter is more than one page long, the
page number will appear at the top of the second and subsequent pages.

\closing{Yours sincerely,}

\end{letter}

}% The brace closes the \applyCSVfile command

\end{document}
